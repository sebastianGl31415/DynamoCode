\section{Structure of the program}
The program run consists of the step initialization, assembling, eigenvalue computation and postprocessing.
The initialization consists of the routines
\begin{compactitem}
	\item \texttt{initializeDiscretization}, which extends the \texttt{params} structure,
	\item \texttt{initializeVelocity}, which extends the \texttt{velocity} structure.
\end{compactitem}
The number of modes is $\texttt{n}_\texttt{modes}=\left(\texttt{nmax}+1\right)^2\,.$ The assembly of the 
matrices consists of the routines
\begin{compactitem}
	\item \texttt{assembleCouplingMatrix}, which consists of the routines
		\begin{compactitem}
			\item \texttt{getPatternMatrix} which computes the \textsc{Wigner}-$3j$-Symbols i.e. the 
			structure \texttt{GeoIntegrals},
			\item \texttt{assembleM} assembles the coupling matrix for toroidal modes 
			$\mathbb{C}^{\left(\texttt{n}_\texttt{modes}\texttt{nr}\right)\times
			\left(\texttt{n}_\texttt{modes}\texttt{nr}\right)}$,
			\item \texttt{assembleN} assembles the coupling matrix for poloidal modes
			$\mathbb{C}^{\left(\texttt{n}_\texttt{modes}\left(\texttt{nr}+1\right)\right)\times
						\left(\texttt{n}_\texttt{modes}\left(\texttt{nr}+1\right)\right)}$,
		\end{compactitem}
	\item \texttt{assembleBesselOperator}, which is rather simple 
	$\mathbb{R}^{\left(\texttt{n}_\texttt{modes}\left(2\texttt{nr}+1\right)\right)\times
							\left(\texttt{n}_\texttt{modes}\left(2\texttt{nr}+1\right)\right)}$,
	\item \texttt{assembleRSquaredMatrix}, which is very simple just a diagonal matrix 
	$\mathbb{R}^{\left(\texttt{n}_\texttt{modes}\left(2\texttt{nr}+1\right)\right)\times
							\left(\texttt{n}_\texttt{modes}\left(2\texttt{nr}+1\right)\right)}$.
\end{compactitem}
The matrix dimension given above indicate that there are two different grids, one for the toroidal modes and 
one for the toroidal modes. The toroidal modes are computed on an \emph{reduced} grid consisting of the 
internal points only. The poloidal modes are computed on an \emph{extended} grid consisting of the internal 
points and the right-hand side boundary point. The eigenvalue computation is performed using the 
\texttt{eigs}-routine, which is a modified \textsc{Arnoldi} method.
\section{Variables}
lists of variables including short descriptions are given below.
\subsection{The structure \texttt{params}}
The structure \texttt{params} contains information related to the discretization and finite differences 
schemes.
\begin{center}
\begin{tabularx}{\textwidth}{rXl}
	variable & description & typ/size\\\hline
	\texttt{diffMethod} & finite difference stencil specification & \texttt{string}\\
	\texttt{nmax} & number of modes & $\mathbb{R}$\\
	\texttt{nr} & number of interior points & $\mathbb{R}$\\
	\texttt{h} & grid spacing & $\mathbb{R}$\\	
	\texttt{rhm} & number of interior points, the reduced grid & $\mathbb{R}^\texttt{nr}$\\
	\texttt{rhn} & number of interior points and right-hand boundary point, the extended grid & 
	$\mathbb{R}^{\texttt{nr}+1}$\\
	\texttt{Rhm} & sparse diagonal matrix with \texttt{rhm} as diagonal entries & 
															$\mathbb{R}^{\texttt{nr}\times\texttt{nr}}$\\
	\texttt{Rhn} & sparse diagonal matrix with \texttt{rhn} as diagonal entries &
								 $\mathbb{R}^{\left(\texttt{nr}+1\right)\times\left(\texttt{nr}+1\right)}$\\
	\texttt{D1m} & sparse FD matrix of 1st derivative on reduced grid &
															$\mathbb{R}^{\texttt{nr}\times\texttt{nr}}$\\
	\texttt{D2m} & sparse FD matrix of 2nd derivative on reduced grid & 
															$\mathbb{R}^{\texttt{nr}\times\texttt{nr}}$\\
	\texttt{D1n} & function handle to get FD Matrix of 1st derivative of poloidal field. The function handle
	has to be called with inputs \texttt{j} according to $n^i_j$ and \texttt{gridType} to specify if result 
	is computed on extended grid or not. & \texttt{function handle}\\
	\texttt{D2n} & function handle to get FD Matrix of 2nd derivative of poloidal field. Same holds for 
	inputs as above.  & \texttt{function handle}\\
	\texttt{D1hm} & Same as \texttt{D1m}. & $\mathbb{R}^{\texttt{nr}\times\texttt{nr}}$\\
	\texttt{D2hm} & Same as \texttt{D2m}. & $\mathbb{R}^{\texttt{nr}\times\texttt{nr}}$\\
	\texttt{D1hn} & sparse FD matrix of 1st derivative on extended grid. This matrix is only applied to 
	velocity fields, which satisfy homogeneous \textsc{Dirichlet} boundary conditions.  & 
								$\mathbb{R}^{\left(\texttt{nr}+1\right)\times\left(\texttt{nr}+1\right)}$\\
	\texttt{D2hn} & sparse FD matrix of 2nd derivative on extended grid. Same as above. & 
								$\mathbb{R}^{\left(\texttt{nr}+1\right)\times\left(\texttt{nr}+1\right)}$
\end{tabularx}
\end{center}
\subsection{The structure \texttt{velocity}}
The structure \texttt{velocity} contains information related to the velocity field. The main of the structure is returned by the routine \texttt{velocity}. The structures \texttt{params} and 
\texttt{velocity.referenceCase} 
need to be specified prior executing \texttt{getPatternMatrix}.
\begin{center}
\begin{tabularx}{\textwidth}{rXl}
	variable & description & typ/size\\\hline
	\texttt{KField} & $K$ integral values and indices stored row-wise in a field. The first three contain the columns contains the index information. The fourth column contains the $K$ integral value. Hence
	\begin{equation*}
		\texttt{KField(x,:)}=
		\begin{bmatrix}
			m/n & i/j & k/l & K^{ikm}_{jln}
		\end{bmatrix}\in\mathbb{R}^{1\times4}\,.
	\end{equation*}
	whereas linear indexing is used for index pairs such as $m/n$.
  & $\mathbb{R}^{\texttt{?}\times\texttt{4}}$\\
	\texttt{LField} & $L$ integral values and indices stored row-wise in a field. Same as above. & 
	$\mathbb{R}^{\texttt{?}\times\texttt{4}}$
\end{tabularx}
\end{center}
\subsection{The structure \texttt{GeoIntegrals}}
The structure \texttt{GeoIntegrals} contains information related to the integrals~$K$ and $L$. The structure 
is returned by the routine \texttt{getPatternMatrix}. The structures \texttt{params} and \texttt{velocity} 
need to be specified prior executing \texttt{getPatternMatrix}.
\begin{center}
\begin{tabularx}{\textwidth}{rXl}
	variable & description & typ/size\\\hline
	\texttt{KField} & $K$ integral values and indices stored row-wise in a field. The first three contain the columns contains the index information. The fourth column contains the $K$ integral value. Hence
	\begin{equation*}
		\texttt{KField(x,:)}=
		\begin{bmatrix}
			m/n & i/j & k/l & K^{ikm}_{jln}
		\end{bmatrix}\in\mathbb{R}^{1\times4}\,.
	\end{equation*}
	whereas linear indexing is used for index pairs such as $m/n$.
  & $\mathbb{R}^{\texttt{?}\times\texttt{4}}$\\
	\texttt{LField} & $L$ integral values and indices stored row-wise in a field. Same as above. & 
	$\mathbb{R}^{\texttt{?}\times\texttt{4}}$
\end{tabularx}
\end{center}

\section{Verification of the Couplingmatrix Assembly}
We choose the velocity field
\begin{equation}
	\bm{v}=\nabla\times\left(r\left(r^2\left(1-r^2\right)\right)
	\alegend[\co{\theta}]{1}{0}\er\right)\,,
\end{equation}
whereas $\texttt{nmax}=1$ und $\texttt{nr}=3$. Hence,
\begin{equation}
	f\left(r\right)=r^2\left(1-r^2\right)\,.
\end{equation}
\section{Implemented Form of Bullard Gellman Equations}
Starting from
\begin{equation}
	\lambda\bm{B}=\Remag\nabla\times\left(\bm{v}\times\bm{B}\right)-\nabla\times\nabla\times\bm{B}
	\DOperator[r]{2}{m^i_j}
\end{equation}
using projection methods we arrive at
\begin{subequations}
\begin{align}
	\lambda j\left(j+1\right)m^i_j\left(r\right)&=j\left(j+1\right)\left(
	\DOperator[r]{2}{m^i_j}- j\left(j+1\right)\frac{m^i_j\left(r\right)}{r^2}	\right)+\Remag\cdots\,.\\
	\lambda j^2\left(j+1\right)^2\frac{n^i_j\left(r\right)}{r^2}&=j^2\left(j+1\right)^2\frac{1}{r^2}\left(
	\DOperator[r]{2}{n^i_j}-j\left(j+1\right)\frac{n^i_j\left(r\right)}{r^2}\right)+\Remag\cdots\,.
\end{align}
\end{subequations}
Multiplying the first equation above $r^2$ and dividing by $j\left(j+1\right)$ leads to
\begin{subequations}
\begin{equation}
	\lambda r^2 m^i_j\left(r\right)=r^2\DOperator[r]{2}{m^i_j}
	-j\left(j+1\right)\frac{m^i_j\left(r\right)}{r^2}+\frac{\Remag}{j\left(j+1\right)}r^2\cdots\,.
\end{equation}
For the second equation after multiplication by $r^4$ and division by $j^2\left(j+1\right)^2$ we obtain
\begin{equation}
	\lambda r^2 n^i_j\left(r\right)=r^2\DOperator[r]{2}{n^i_j}
	-j\left(j+1\right)n^i_j\left(r\right)+\frac{\Remag}{j^2\left(j+1\right)^2}r^4\cdots\,.
\end{equation}
\end{subequations}
For radial differential operator it holds
\begin{equation}
	r^2\DOperator[r]{2}{f}
	=r\ddsqr{}{r}\left(rf\left(r\right)\right)=r^2f''\left(r\right)+2rf'\left(r\right)\,.
\end{equation}
Hence, the implemented equations are of the form
\begin{subequations}
\begin{align}
	\lambda r^2 m^i_j\left(r\right)&=r^2\ddsqr{m^i_j}{r}+2r\dd{m^i_j}{r}
	- j\left(j+1\right)m^i_j\left(r\right)+\frac{\Remag}{j\left(j+1\right)}r^2\cdots\,.\\
	\lambda r^2 n^i_j\left(r\right)&=r^2\ddsqr{n^i_j}{r}+2r\dd{n^i_j}{r}
	-j\left(j+1\right)n^i_j\left(r\right)+\frac{\Remag}{j^2\left(j+1\right)^2}r^4\cdots\,.
\end{align}
\end{subequations}