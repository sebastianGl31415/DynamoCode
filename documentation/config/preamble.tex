%------------------------------------------------------------------------------%
%------------------------------------------------------------------------------%
%--- Titel:    ...                                                          ---%
%--- Autor:    ...                                                          ---%
%--- Vorlage:  Dipl.-Ing. Timo Pe <timo.pe@tu-harburg.de>                   ---%
%---           Institut für Flugzeug-Systemtechnik                          ---%
%---           Neßpriel 5, 21129 Hamburg                                    ---%
%---           http://www.fst.tu-harburg.de/                                ---%
%------------------------------------------------------------------------------%
%------------------------------------------------------------------------------%


%------------------------------------------------------------------------------%
%---- Hinweise ----------------------------------------------------------------%
%------------------------------------------------------------------------------%

% Folgende Abkürzungen dienen nur der besseren Lesbarkeit in den Kommentaren:
% $miktex$ = Pfad zum Hauptverzeichnis der MiKTeX-Installation
% $groups$ = http://groups.google.de/group



%------------------------------------------------------------------------------%
%--- \listfiles ---------------------------------------------------------------%
%------------------------------------------------------------------------------%

% >> \listfiles
% Funktion: Ausgabe der verwendeten Pakete in der Log-Datei
% Hinweise: Durch die Log-Datei ist im Nachhinein nachvollziehbar, welche
%           Paketversionen verwendet wurden. Der Name der Log-Datei wird durch
%           das Hauptdokument (\jobname, hier: fst-template) bestimmt:
%           \jobname.log
% Gebrauch: \listfiles
% Optionen: -
% vor/nach: vor \documentclass{}
% Referenz: $miktex$/doc/latex/base/usrguide.dvi
\listfiles



%------------------------------------------------------------------------------%
%---- Dokumentenklasse --------------------------------------------------------%
%------------------------------------------------------------------------------%

% >> \documentclass
% Funktion: Dokumentenklasse definieren
% Hinweise: Verwendet werden die Klassen aus dem KOMA-Script
% Gebrauch: -
% Optionen: siehe scrguide.pdf
% vor/nach: -
% Referenz: $miktex$/doc/latex/koma-script/scrguide.pdf
% KOMA-Script < 3.00 (erzeugt Warnings für >= 3.00)
% \documentclass[%
%     paper=a4,               % Papierformat
%     twoside,                % beidseitig bedruckt
%     fontsize=12pt,          % Schriftgröße
%     titlepage,              % Titelseite
%     liststotoc,             % Tabellen-/Abb.verzeichnis ins Inhaltsverzeichnis
%     bibtotoc,               % Literaturverzeichnis ins Inhaltsverzeichnis
%     listsleft,              % Tabellen-/Abb.verzeichnis ohne Einzug
%     openright,              % Kapitel beginnen immer auf der rechten Seite
%     cleardoublepage=empty,  % pagestyle für leere Seite vor Kapitel
%     headsepline=true,       % Header mit Linie abtrennen
%     footsepline=false,      % Footer nicht mit Linie abtrennen
%     tablecaptionabove,      % Abstände anpassen, für Captions oberhalb von Tab.
%     pointlessnumbers,       % sonst bei Verwendung von \appendix autom. Punkte
%     halfparskip-,           % halber Zeilenabstand zwischen Absätzen
%     normalheadings          % Überschriften "normaler" Größe
% ]{scrbook}

% KOMA-Script >= 3.00 (Optionen nach key=value-Prinzip)
\documentclass[%
    paper=A4,               % Papierformat
    twoside=false,           % beidseitig bedruckt
    fontsize=12pt,          % Schriftgröße
    titlepage=false,         % Titelseite
    listof=totoc,           % Tabellen-/Abb.verzeichnis ins Inhaltsverzeichnis
    bibliography=totoc,     % Literaturverzeichnis ins Inhaltsverzeichnis
    listof=flat,            % Tabellen-/Abb.verzeichnis ohne Einzug
%    open=right,             % Kapitel rechts beginnen
    cleardoublepage=empty,  % pagestyle für leere Seite vor Kapitel
    headsepline=true,       % Header mit Linie abtrennen
    footsepline=false,      % Footer nicht mit Linie abtrennen
    captions=tableheading,	% Abstände anpassen, für Captions oberhalb von Tab.
    numbers=noendperiod,    % keine Punkte am Ende von Kapitel-/Anhangnummern
    parskip=half-,          % halber Zeilenabstand zwischen Absätzen
    headings=normal,         % Überschriften "normaler" Größe
%    draft
]{scrartcl}					% Version 3.00 wurde 2008-11-03 veröffentlicht


% >> scrhack
% Funktion: KOMA-Paket, das Probleme mit _fremden_ Paketen löst
% Hinweise: Probleme mit hyperref, listings, float werden adressiert
% Gebrauch: -
% Optionen: -
% vor/nach: -
% Referenz: $miktex$/doc/latex/koma-script/scrhack.pdf
\usepackage{scrhack}
\KOMAoptions{hyperref=false}



%------------------------------------------------------------------------------%
%---- Schrifttyp --------------------------------------------------------------%
%------------------------------------------------------------------------------%

% >> lmodern
% Funktion: Latin Modern Schriftart
% Hinweise: wie Computer Modern Schriftart, jedoch ist die "bold"-Variante
%           fetter, außerdem besser Unterstützung von Akzenten etc.
% Gebrauch: -
% Optionen: -
% vor/nach: Schrifttyp vor der Satzspiegelberechnung durch typearea definieren
% Referenz: $miktex$/doc/fonts/lm/lm-info.pdf
\usepackage{lmodern}


% >> inputenc
% Funktion: Umlaute etc. direkt im Quelltext ermöglichen
% Hinweise: -
% Gebrauch: -
% Optionen: latin1: ISO Latin Zeichen
% vor/nach: -
% Referenz: $miktex$/doc/latex/base/inputenc.dvi
\usepackage[utf8]{inputenc}


% >> lccode/defaulthyphenchar
% Funktion: Silbentrennung auch bei Wörtern, die Bindestrich besitzen
% Hinweise: Bei zusammengesetzten Wörter, die einen Bindestrich besitzen
%           (z.B. Flugzeug-Systemtechnik) wird standardmäßig NUR am Bindestrich
%           getrennt. Vor oder nach dem Bindestrich wird nicht getrennt.
%           Insbesondere bei langen Wörter, kann es so schnell zu underfull oder
%           overfull boxes kommen.
% Gebrauch: -
% Optionen: -
% vor/nach: fontenc
% Referenz: $groups$/de.comp.text.tex/msg/35c3777fc9a6fd6b
% Referenz: $groups$/de.comp.text.tex/msg/1984119fb99db691
% Referenz: $groups$/de.comp.text.tex/browse_thread/thread/b629ea667f5b7045
\lccode`\-=`\-
\defaulthyphenchar=127 


% >> fontenc
% Funktion: Zeichentabelle für Sonderzeichen/Umlaute erweitern
% Hinweise: Umlaute werden (trotz lmodern) standardmäßig aus normalen Zeichen
%           plus Punkte/Akzente etc. zusammengesetzt. fontenc ermöglicht
%           eine erweiterte Zeichentabelle mit expliziten Sonderzeichen etc.
% Gebrauch: -
% Optionen: T1: Font-Kodierung für westeuropäische Sprachen
% vor/nach: -
% Referenz: -
\usepackage[T1]{fontenc}



%------------------------------------------------------------------------------%
%---- Satzspiegelberechnung ---------------------------------------------------%
%------------------------------------------------------------------------------%

% Für Infos zur Satzspiegelberechnung siehe:
% - $miktex$/doc/latex/koma-script/scrguide.pdf
% - "The LaTeX Companion", S. 194, 2nd. edition (2004)
% - $miktex$/doc/latex/geometry/geometry.dvi

% Hinweis:
% Die Vorlage wählt bewusst eine feste Größe des Textbereiches, der sowohl in
% der Druck- als auch in der Digitalversion gleich bleibt. So werden
% Verschiebungen verhindert. Außerdem hat es den Vorteil, dass der Autor für
% Bilder mit einer maximalen Größe rechnen kann.

% >> geometry
% Funktion: Satzspiegelberechnung durch Vorgabe fixer Parameter
% Hinweise: Die offenen Parameter werden vom Paket automatisch berechnet
% Optionen: siehe $miktex$/doc/latex/geometry/geometry.dvi
% vor/nach: vor fancyhdr
% Referenz: $miktex$/doc/latex/geometry/geometry.dvi
\usepackage[a4paper]{geometry}

% Mit der if-Abfrage \ifprint wird zwischen Druck- und Digitalversion
% umgeschaltet!
\ifprint % Druck: Ränder Doppelseite im Verhältnis 1:1:1, mit Bindekorrektur
  \geometry{%
    includehead=true, % wg. fancyhdr-Formatierung zählt Head optisch zum Text
    hmarginratio=1:2, % Rand innen:außen (für Doppelseite ergibt sich: 1:1:1)
    vmarginratio=3:5, % Rand oben:unten (Standard: typearea 1:2, geometry 2:3)
    textwidth=170mm,  % Breite der Textfläche (ohne Randnotizen)
    textheight=230mm, % Höhe der Textfläche (ohne Header)
    headheight=20pt,  % Header größer gegen overfull vboxes (Standard: 12pt)
    footskip=15mm,    % Abstand Baseline des Footer zum Textkörper
    bindingoffset=6mm % Bindekorrektur BCOR
  }%
\else % Digital: horizontal zentriert, ohne Bindekorrektur
  \geometry{%
    includehead=true, % wg. fancyhdr-Formatierung zählt Head optisch zum Text
    hmarginratio=1:1, % Rand innen:außen (PDF: Textkörper zentrieren!)
    vmarginratio=3:5, % Rand oben:unten (Standard: typearea 1:2, geometry 2:3)
    textwidth=170mm,  % Breite der Textfläche (ohne Randnotizen)
    textheight=230mm, % Höhe der Textfläche (ohne Header)
    headheight=20pt,  % Header größer gegen overfull vboxes (Standard: 12pt)
    footskip=15mm,    % Abstand Baseline des Footer zum Textkörper
    bindingoffset=0mm % Bindekorrektur BCOR
  }%
\fi



%------------------------------------------------------------------------------%
%---- Absatzlayout ------------------------------------------------------------%
%------------------------------------------------------------------------------%

% >> setspace
% Funktion: Zeilenabstand verändern
% Hinweise: global oder lokal beschränkt
% Gebrauch: -
% Optionen: -
% vor/nach: wenn global: vor typearea
% Referenz: $miktex$/tex/latex/setspace/setspace.sty
\usepackage{setspace}


% >> hanging
% Funktion: Hängender Einzug
% Hinweise: -
% Gebrauch: -
% Optionen: -
% vor/nach: -
% Referenz: $miktex$/doc/latex/hanging/hanging.pdf
%\usepackage{hanging}


% >> ...penalty
% Funktion: Alleinstehende Zeilen verhindern
% Hinweise: Durch penalty werden alleinstehende Zeilen "bestraft".
%           Durch einen hohen Wert (10000) ist die Bestrafung maximal,
%           so dass die Zeilen verhindert werden
% Gebrauch: \widowpenalty = 10000 (alleinstehende Zeilen am Seitenkopf)
%           \clubpenalty = 10000 (alleinstehende Zeilen am Seitenfuß)
%           \displaywidowpenalty = 10000 (alleinstehende Zeilen nach Formeln)
% Optionen: -
% vor/nach: -
% Referenz: -
\widowpenalty=10000
\clubpenalty=10000
\displaywidowpenalty=10000


% >> moderate Umgebung
% Funktion: Einstellungen der Umbruchautomatiken
% Hinweise: Axel Reichert "moderate" Werte, die einen guten Kompromiss für
%           Umbruchprobleme (underfull- und overfull-boxes) darstellen.
% Gebrauch: \tolerance 1414
%           \emergencystretch 1.5em
%           \hbadness 1414
%           \hfuzz 0.3pt
%           \vfuzz 0.3pt
% Optionen: -
% vor/nach: -
% Referenz: $groups$/de.comp.text.tex/msg/c375ef11e78e7bfa


% >> \tolerance und \emergencystretch
% Funktion: Zeilenumbruchautomatiken
% Hinweise: TeX’s first attempt at breaking lines is performed without even
%           trying hyphenation: TeX sets its “tolerance” of line breaking
%           oddities to the internal value \pretolerance, and sees what happens.
%           If it can’t get an acceptable break, TeX adds the hyphenation points
%           allowed by the current patterns, and tries again using the internal
%           \tolerance value. If this pass also fails, and the internal
%           \emergencystretch value is positive, TeX will try a pass that allows
%           \emergencystretch worth of extra stretchability to the spaces in
%           each line.
% Gebrauch: -
% Optionen: -
% vor/nach: -
% Referenz: http://www.tex.ac.uk/cgi-bin/texfaq2html?label=overfull
\tolerance 1414
\emergencystretch 1.5em


% >> \hbadness
% Funktion: -
% Hinweise: Grenzwert für »schlechte« Zeilen, bzw. Boxen, der angibt, ab
%           welcher Negativbewertung eine \hbox protokolliert wird. 
% Gebrauch: -
% Optionen: -
% vor/nach: -
% Referenz: www.jr-x.de/publikationen/latex/tipps/zeilenumbruch.html
\hbadness 1414


% >> \hfuzz \vfuzz
% Funktion: Horizontalen/vertikalen Überstand erlauben, ohne overfull zu melden
% Hinweise: -
% Gebrauch: \hfuzz [Länge]
%           \vfuzz [Länge]
% Optionen: -
% vor/nach: -
% Referenz: -
\hfuzz 0.3pt
\vfuzz 0.3pt


% >> \raggedbottom \flushbottom
% Funktion: Bestimmt wo der letzte Absatz platziert wird
% Hinweise: -
% Gebrauch: \raggedbottom (Voreinstellung für article, report und letter o.ä.)
%           > keine Dehnung der Absätze zum Textfuß, d.h. der letzte
%           Absatz Ende u.U. schon vor dem Textfuß
%           \flushbottom (Voreinstellung für book o.ä.)
%           > letzter Absatz endet immer am Textfuß, d.h. Dehnung
% Optionen: -
% vor/nach: -
% Referenz: http://www.weinelt.de/latex/raggedbottom.html
%           http://www.weinelt.de/latex/flushbottom.html
\raggedbottom
%\flushbottom



%------------------------------------------------------------------------------%
%---- Abschnittslayout (chapter, sections, paragraphs) ------------------------%
%------------------------------------------------------------------------------%

% >> \...depth
% Funktion: Tiefe der Überschriftennummerierung/-anzeige
% Hinweise: part:         -1
%           chapter:       0
%           section:       1
%           subsection:    2
%           subsubsection: 3
%           paragraph:     4
%           subparagraph:  5
% Gebrauch: secnumdepth: Tiefe der im Text zu nummerierenden Überschriften
%           tocdepth: Tiefe der im Inhaltsverzeichnis angezeigten Überschriften
% Optionen: -
% vor/nach: -
% Referenz: $miktex$/doc/latex/koma-script/scrguide.pdf
%           DE-TeX-FAQ, Abschnitt 7.1.3, http://www.dante.de/faq/de-tex-faq/
%\setcounter{secnumdepth}{2} % subsection
%\setcounter{tocdepth}{2} % subsection


% Layout von sections/paragraphs ändern
% Die Änderung erfolgt manuell durch Kopieren einer vorhandenen Formatierung
% und manuelles Anpassen. Die Layouts können aus den entsprechenden
% Dokumentenklassen kopiert werden.
%
%\@startsection werden folgende Parameter übergeben:
% #1 Name der Gliederungsebene
% #2 Tiefe der Ebene (toclevel)
% #3 Einzug vom linken Rand
% #4 Abstand zum vorhergehenden Text (beforeskip):
%    - fester Abstand: z.B. "2.5ex"
%    - dehnbarer Abstand (glue): z.B. "2.5ex \@plus0.6ex \@minus0.15ex"
%    - negative Werte: Unterdrückung des Einzugs des folgenden Abstands (?)
% #5 Abstand nach der Überschrift (afterskip):
%    - wie bei #4, allerdings:
%    - bei negativen Werten wird die Überschrift eingebettet
%     (wie es bei paragraph voreingestellt ist)
% #6 Formatierung der Schrift:
%    z.B. \bfseries, \large etc.
%
% Hinweis:
% - \p@ = Token für "1pt"
% - \z@ = Token für "0pt"

% >> \renewcommand\paragraph
% Funktion: Punkt hinter \paragraph-Überschrift einfügen
% Hinweise: Kopie von Paragraph aus scrbook.cls, \dottedsection angefügt.
% Gebrauch: -
% Optionen: -
% vor/nach: -
% Referenz: $groups$/de.comp.text.tex/msg/60942741d3ede7d7

%\newcommand{\dottedsection}[1]{#1.} 
%
%\makeatletter
%\renewcommand\paragraph{%
%\@startsection%
%  {paragraph}%
%  {4}%
%  {\z@}%
%  {3.25ex \@plus1ex \@minus.2ex}%
%  {-1em}%
%  {\raggedsection\normalfont\sectfont\nobreak\size@paragraph\dottedsection}%
%}
%\makeatother


% >> \@pnumwidth
% Funktion: erhöht die Breite für die Seitenzahlen im TOC
% Hinweise: bei sehr breiten Zahlen >1000 oder römische Zahlen reicht der 
%           Standardwert von 1.5em nicht aus, so dass overfull hboxes kommen.
% Gebrauch: auch nur für einzelne Abschnitte hin- und herschalten:
%           \addtocontents{toc}{\protect\renewcommand{\protect\@pnumwidth}{3em}}
% Optionen: -
% vor/nach: -
% Referenz: DE-TeX-FAQ, Abschnitt 7.1.1, http://www.dante.de/faq/de-tex-faq/
%\makeatletter
%\renewcommand{\@pnumwidth}{1.85em} % Standard: 1.5em
%\makeatother


% >> \renewcommand{\chapter}
% Funktion: Chapter in den PDF-Bookmarks fett darstellen
% Hinweise: -
% Gebrauch: -
% Optionen: -
% vor/nach: -
% Referenz: $groups$/de.comp.text.tex/browse_thread/thread/9a3956b262a12490
%\makeatletter
%\newcommand{\saved@chapter}{}
%\let\saved@chapter\chapter
%\newcommand*{\my@chapter}[2][]{%
%  \bookmarksetup{bold=true}%
%  \saved@chapter[#1]{#2}%
%  \bookmarksetup{bold=false}%
%}
%
%\renewcommand{\chapter}{%
%  \@ifstar{\saved@chapter*}{\@dblarg\my@chapter}%
%}
%\makeatother


% >> \chapterheadstartvskip
% Funktion: Abstand Seitenanfang <> Kapitelüberschrift
% Hinweise: Im KOMA-Script ist standardmäßig ein relativ großer Abstand
%           zwischen Seitenanfang und Kapitelüberschrift definiert.
%           Achtung: definiert man \chapterheadstartvskip leer, dann bleibt
%           trotzdem noch ein Abstand. Um diesen ganz zu entfernen muss man
%           mit negativen Längen arbeiten: \vspace*{-5\topskip}
% Gebrauch: -
% Optionen: -
% vor/nach: -
% Referenz: $groups$/de.comp.text.tex/browse_thread/thread/48eb96b95c1f7bf3/
%\renewcommand*{\chapterheadstartvskip}{%
%\vspace*{\baselineskip} % KOMA-Script-Standard: 2.3\baselineskip
%}

\usepackage{scrpage2}

%------------------------------------------------------------------------------%
%---- Mathematisches ----------------------------------------------------------%
%------------------------------------------------------------------------------%

% >> icomma
% Funktion: Komma als Dezimaltrennzeichen im Mathematik-Modus
% Hinweise: Standardmäßig ist der Punkt das Dezimaltrennzeichen und das Komma
%           nur für Auflistungen (z.B. f(a, b)) gedacht. Das Komma erzeugt
%           deshalb einen zu großen Abstand.
% Gebrauch: -
% Optionen: -
% vor/nach: -
% Referenz: $miktex$/doc/latex/was/icomma.dvi
%           http://projekte.dante.de/DanteFAQ/MathematischerFormelsatz#25
%\usepackage{icomma}


% >> amsmath
% Funktion: Mathematik-Umgebungen/-Befehle
% Hinweise: Standardmäßig sind alle griechischen Großbuchstaben bei AMS
%           aufrecht, da sie meist Operatoren sind. Für die kursive Variante
%           (als Variablen) steht \var... zur Verfügung: \Delta \varDelta
% Gebrauch: -
% Optionen: sumlimits: Grenzen bei Summenzeichen ober-/unterhalb (statt daneben)
%           intlimits: Grenzen bei Integralen ober-/unterhalb (statt daneben)
%           sum-/intlimits und sqrt{} > Wurzelzeichen wird gerade dargestellt,
%           weil die Zeichen dadurch zu hoch werden
% vor/nach: vor \numberwithin (wird nach hyperref verwendet, s.u.)
% Referenz: $miktex$/doc/latex/amsmath/amsldoc.dvi
\usepackage[sumlimits,intlimits]{amsmath}
\usepackage{amssymb}
\usepackage{amsthm}
\usepackage{amsxtra}
\usepackage{amsfonts}
%\usepackage{mathrsfs}
%\usepackage{eucal}
\usepackage{stackrel}
\usepackage{tensor}
\usepackage{mathtools}
\usepackage{esint}
%\usepackage{cancel}

% >> easybmat
% Funktion: Matrizen mit Linien/gleichen Spalten-/Reihengrößen
% Hinweise: Alternativ auch arydshln für arrays verwendbar
% Gebrauch: -
% Optionen: -
% vor/nach: -
% Referenz: $miktex$/doc/latex/easy/docbmat.dvi
%\usepackage{easybmat}


% >> bm, amsbsy
% Funktion: Fettschreibung (mathematischer) Zeichen
% Hinweise: amsbsy: AMS poor mans bold (Zeichen wird verdoppelt)
% Gebrauch: \bm{Zeichen} \pmb{Zeichen}
% Optionen: -
% vor/nach: vor ngerman, da es sonst Konflikte gibt, siehe auch:
%           $groups$/de.comp.text.tex/browse_thread/thread/8367cd3f38280631
%           $groups$/de.comp.text.tex/msg/da31045a7a78352b
% Referenz: $miktex$/doc/latex/tools/bm.dvi
%           $miktex$/doc/latex/amsmath/amsldoc.dvi
\usepackage{bm}
%\usepackage{amsbsy}



%------------------------------------------------------------------------------%
%---- Rechtschreibung, Silbentrennung -----------------------------------------%
%------------------------------------------------------------------------------%

% >> hyphsubst mit dehyph-exptl Trennmuster
% Funktion: Verbesserung der Trennmuster (Silbentrennung)
% Hinweise: -
% Gebrauch: -
% Optionen: -
% vor/nach: vor babel
% Referenz: $miktex$/doc/generic/dehyph-exptl/dehyph-exptl.pdf
%\RequirePackage[ngerman=ngerman-x-latest]{hyphsubst}


% >> babel
% Funktion: Silbentrennung, lokale Namen
% Hinweise: -
% Gebrauch: -
% Optionen: ngerman: neue deutsche Rechtschreibung
%           english: englische Rechtschreibung
% vor/nach: -
% Referenz: $miktex$/doc/generic/babel/babel.pdf
\usepackage[english]{babel}
%\usepackage[english]{babel}


% >> ngerman
% Funktion: diverse Makros für die Vorgabe der Silbentrennung
% Hinweise: When babel is used with the options ngerman or naustrian,
%           babel will provide all features of the package ngerman.
% Gebrauch: -
% Optionen: -
% vor/nach: -
% Referenz: $miktex$/doc/latex/german/gerdoc.dvi
%\usepackage{ngerman}


% >> ragged2e
% Funktion: Verbessert den Flattersatz
% Hinweise: Inbesondere bei kurzen Spalten/Textbreiten besserer Flattersatz
% Gebrauch: \RaggedRight (=\FlushLeft)
%           \RaggedLeft (=\FlushRight)
%           > \Centering nicht verwendet, erzeugt underfull boxes,
%             stattdessen den Standardbefehl \centering
% Optionen: -
% vor/nach: -
% Referenz: $miktex$/doc/latex/ms/ragged2e.dvi
\usepackage{ragged2e}


% >> \left- und \righthyphenmin
% Funktion: Mindestanzahl von Zeichen links/rechts neben der Trennstelle
% Hinweise: Standard für beide Kommandos: 2
% Gebrauch: -
% Optionen: -
% vor/nach: -
% Referenz: -
%\lefthyphenmin=3 % Standard: 2
%\righthyphenmin=3 % Standard: 2



%------------------------------------------------------------------------------%
%---- Einheiten ---------------------------------------------------------------%
%------------------------------------------------------------------------------%

% Diskussion über zu verwendendes Paket:
% $groups$/de.comp.text.tex/browse_thread/thread/52817543ec82809a
% $groups$/de.comp.text.tex/browse_thread/thread/8f1fa8b938541600
%
% Wichtiges über Schreibweisen, Formatierung der Einheiten:
% http://www.ptb.de/de/publikationen/download/si_v1.pdf
% http://physics.nist.gov/Pubs/SP330/sp330.pdf

% >> units/sistyle/SIunits/fancyunits/siunitx etc.
% Funktion: Einheitenformatierung
% Hinweise: siehe unten
% Gebrauch: units: \unit[#1]{#2}
%           sistyle: \SI{#1}{#2}
% Optionen: -
% vor/nach: -
% Referenz: $miktex$/doc/latex/units/units.dvi
%           $miktex$/doc/latex/sistyle/SIstyle-2.x.pdf
%           $miktex$/doc/latex/siunits/SIunits.pdf
%           fancyunits.pdf
%           $miktex$/doc/latex/unitsdef/unitsdef.pdf
%\usepackage{units}      % nur Abstand
%\usepackage{SIunits}    % Einheitenmakros, besser: fancyunits
%\usepackage{fancyunits} % basiert auf SIunits, verbessert bzgl. Schrift
%\usepackage{siunitx}    % Einheitenmakros, Nummernformatierung
\usepackage{sistyle}     % Nummernformatierung

% Produktzeichen \SI{5e2}{N} wird zu: $5 \cdot 10^2\,\mathrm{N}$
\SIproductsign{\cdot} 

% Dezimaltrennzeichen: Komma
\SIdecimalsign{,}


% Hinweise zur Auswahl des Einheiten-Pakets:
% - units kümmert sich nur um den Abstand
% - \unit[#1]{#2}:
%   #1: kann beliebiger Text sein
%   #2: kann beliebiger Text sein
% - sistyle hat den Vorteil, dass auch die Nummern formatiert werden
%   (Gruppierung, Dezimalstellen etc.), Schreibweise schnell und übersichtlich.
%   Keine vordefinierten Befehle, können aber in Verbindung mit SIunits
%   trotzdem genutzt werden!
%   Nachteil: bei Klammerung oder anderen Sonderzeichen wird trotzdem
%   Gruppierung versucht!
% - sistyle \SI{#1}{#2}:
%   #1: Mathemodus, Gruppierung zu 3er als Standard
%   #2: eigene Syntax, siehe manual
% - SIunits scheint etwas zu aufwendig bezüglich der Schreibweise, dafür aber
%   konsistenter in der Formatierung, wenn man die vordefinierten Befehle
%   verwendet
% - SIunits \unit{#1}{#2}:
%   #1: Mathemodus
%   #2: Mathemodus
% - sistyle wird nicht mehr weiterentwickelt > Verweis auf siunitx
% - siunitx: sehr gutes Paket, mit Einheitenmakros, Nummernformatierung und
%   umfangreichen Konfigruationsmöglichkeiten
%   > sistyle scheint aber intuitiver, da Einheiten direkt als Zeichen
%     eingegeben werden können und auch \frac in der Einheit problemlos verwendet
%     werden kann.
%
% >> SIstyle wird verwendet:
%    \SI{1}{mm} \SI{123456.12E-8}{\frac{kg}{s}}
%    Wenn Zeichen, die nicht Zahlen sind, vorkommen, manuell setzen:
%    (255/255/170)\,mm (2\,--\,10)\,mm



%------------------------------------------------------------------------------%
%---- Längen ------------------------------------------------------------------%
%------------------------------------------------------------------------------%

% >> calc
% Funktion: Längenberechnungen etc.
% Hinweise: -
% Gebrauch: -
% Optionen: -
% vor/nach: -
% Referenz: $miktex$/doc/latex/tools/calc.dvi
%\usepackage{calc}


% >> printlen
% Funktion: Umrechnung der Längen: pt,mm,cm,in, ...
% Hinweise: -
% Gebrauch: \uselengthunit{mm}\printlength{\textwidth}
% Optionen: -
% vor/nach: -
% Referenz: $miktex$/tex/latex/ltxmisc/printlen.sty
%\usepackage{printlen}



%------------------------------------------------------------------------------%
%---- Tabellen/Arrays ---------------------------------------------------------%
%------------------------------------------------------------------------------%

% >> array
% Funktion: Verbesserung von Tabellen und Arrays
% Hinweise: -
% Gebrauch: -
% Optionen: -
% vor/nach: -
% Referenz: $miktex$/doc/latex/tools/array.dvi
\usepackage{array}


% >> multirow
% Funktion: Verbinden von Zellen in einer Spalte
% Hinweise: -
% Gebrauch: \multirow{<Anzahl der Zeilen>}{"*" oder <Breite>}{<Text>}
% Optionen: -
% vor/nach: -
% Referenz: $miktex$/doc/faq/english/FAQ-multirow.html
\usepackage{multirow}


% >> tabularx
% Funktion: Tabellenbreite passt sich der Gesamtbreite an
% Hinweise: Neuer Spalentyp "X"
%           Spaltentyp L/R/C wird in macros.tex definiert
% Gebrauch: -
% Optionen: -
% vor/nach: -
% Referenz: $miktex$/doc/latex/tools/tabularx.dvi
%           tabsatz.pdf
\usepackage{tabularx}


% >> booktabs
% Funktion: Tabellen ohne vertikale Linien
% Hinweise: -
% Gebrauch: \toprule \midrule \bottomrule
%           \cmidrule(<l/r/lr>)
% Optionen: -
% vor/nach: -
% Referenz: $miktex$/doc/latex/booktabs/booktabs.dvi
%           auch: tabsatz.pdf
\usepackage{booktabs}


% >> arydshln
% Funktion: Gestrichelte Linien etc. in Arrays/Matrizen
% Hinweise: -
% Gebrauch: -
% Optionen: -
% vor/nach: nach array, longtable, colortab, colortbl
% Referenz: $miktex$/doc/latex/arydshln/aryshln-man.pdf
%\usepackage{arydshln}


% >> dcolumn/rccol
% Funktion: Tabellenzellen am Dezimaltrennzeichen ausrichten
% Hinweise: dcolumn: nur Ausrichten
%           rccol: Runden und Ausrichten
% Gebrauch: dcolumn: D{sep.tex}{sep.dvi}{decimal places}
%                    ggfs. über \newcolumntype definieren!
%                    Wenn keine Ausrichtung: \multicolumn{1}{c/l/r}{...}
% Optionen: -
% vor/nach: -
% Referenz: $miktex$doc/latex/tools/dcolumn.dvi
%           $miktex$doc/latex/rccol/rccol.dvi
%\usepackage{dcolumn}
%\usepackage{fltpoint} % Benötigt von rccol Package
%\usepackage{rccol}


% >> colortbl
% Funktion: Farbige Tabellen
% Hinweise: -
% Gebrauch: -
% Optionen: -
% vor/nach: -
% Referenz: -
\usepackage{colortbl}

%------------------------------------------------------------------------------%
%---- Listen, Aufzählungen ----------------------------------------------------%
%------------------------------------------------------------------------------%

% >> paralist
% Funktion: Formatierung von Listen/Aufzählungen
% Hinweise: Formatierung der Listen in config/docspecific.tex
% Gebrauch: -
% Optionen: neverdecrease ist notwendig, damit bei Angabe eines alternativen
%           Labels der leftmargin weiterhin verwendet wird (sonst ohne Abstand)
% vor/nach: -
% Referenz: $miktex$/doc/latex/paralist/paralist.dvi
%           "The LaTeX Companion", S. 145, 2nd. edition (2004)
\usepackage[neverdecrease]{paralist}



%------------------------------------------------------------------------------%
%---- Floats ------------------------------------------------------------------%
%------------------------------------------------------------------------------%

% >> top-/bottom-/totalnumber
% Funktion: Maximale Anzahl der floats
% Hinweise: Maximale Anzahl der floats, die auf einer Seite top bzw. bottom
%           bzw. insgesamt auf der Seite platziert werden.
% Gebrauch: -
% Optionen: -
% vor/nach: -
% Referenz: DE-TeX-FAQ, Abschnitt 6.1.3, http://www.dante.de/faq/de-tex-faq/
\setcounter{topnumber}{3}     % Standard: 2
\setcounter{bottomnumber}{2}  % Standard: 1
\setcounter{totalnumber}{4}   % Standard: 3


% >> \...fraction
% Funktion: Maximaler zur Verfügung stehender Anteil bzw. mindestens
%           ausgefüllter Anteil
% Hinweise: 0 = 0%; 1 = 100%
% Gebrauch: -
% Optionen: -
% vor/nach: -
% Referenz: DE-TeX-FAQ, Abschnitt 6.1.3, http://www.dante.de/faq/de-tex-faq/

% Mindestanteil zur Erzeugung einer float-page (nur floats auf einer Seite)
\renewcommand{\floatpagefraction}{0.7}  % Standard: 0.5

% Mindestanteil von Text, sonst wird die Seite zur float-page
\renewcommand{\textfraction}{0.1}       % Standard: 0.2

% Maximalanteil, den floats am Seitenanfang einer Seite einnehmen dürfen.
\renewcommand{\topfraction}{0.9}        % Standard: 0.7

% Maximalanteil, den floats am Seitenende einer Seite einnehmen dürfen.
\renewcommand{\bottomfraction}{0.7}     % Standard: 0.3


% >> Float Platzierung
% Funktion: Default-Platzierungen von Gleitobjekten ändern.
% Hinweise: Standard war "tbp" (top,bottom,page).
%           Geändert zu "htbp" (here,top,bottom,page)
% Gebrauch: -
% Optionen: -
% vor/nach: -
% Referenz: DE-TeX-FAQ, Abschnitt 6.1.2, http://www.dante.de/faq/de-tex-faq/
\makeatletter
\renewcommand{\fps@figure}{htbp}%
\renewcommand{\fps@table}{htbp}%
\makeatother


% >> floats auf reinen float pages ausrichten
% Funktion: Vertikale Ausrichtung von floats auf float pages
% Hinweise: Deaktiviert, wegen falscher Platzierung bei sidewaysfigure im
%           rotating-Paket!
% Gebrauch: \setlength{\@fptop}{0pt} : oberer Seitenrand
%           Standard: mittig
% Optionen: -
% vor/nach: -
% Referenz: DE-TeX-FAQ, Abschnitt 6.1.18, http://www.dante.de/faq/de-tex-faq/
% \makeatletter
% \setlength{\@fptop}{0pt}%
% \makeatother


% >> float
% Funktion: Mehr Funktionen für Floats
% Hinweise: neue Option [H] bei figures, platziert Grafik genau an der
%           gewählten Stelle; sollte nur in Ausnahmefällen verwendet werden
% Gebrauch: -
% Optionen: -
% vor/nach: -
% Referenz: $miktex$/doc/latex/float/float.dvi
\usepackage{float}



%------------------------------------------------------------------------------%
%---- Farben und Grafiken -----------------------------------------------------%
%------------------------------------------------------------------------------%

% >> xcolor
% Funktion: Farben definieren
% Hinweise: -
% Gebrauch: -
% Optionen: dvips: Treiber für dvi, pdftex: Treiber für pdf
% vor/nach: vor Farbdefinitionen von hyperref
% Referenz: $miktex$/doc/latex/xcolor/xcolor.pdf
%           auch: grfguide.ps
%\usepackage[pdftex]{xcolor}
\usepackage{xcolor}


% >> graphicx
% Funktion: Einbindung von Grafiken
% Hinweise: -
% Gebrauch: -
% Optionen: dvips: Treiber für dvi (eps-Grafiken)
%           pdftex: Treiber für pdf (keine eps-Grafiken, dafür jpg etc.)
% vor/nach: -
% Referenz: $miktex$doc/latex/graphics/graphicx.dvi
%           $miktex$doc/latex/graphics/grfguide.pdf
%\usepackage[dvips]{graphicx}
%\usepackage[pdftex]{graphicx}
\usepackage{graphicx}


% >> epstopdf
% Funktion: eps-Bilder automatisch zu pdf-Bilder wandeln
% Hinweise: Werden bei Verwendung von pdflatex eps-Bilder gefunden, so werden
%           diese automatisch umgewandelt.
%           Achtung pdflatex muss mit "--enable-write18" (MiKTeX) oder
%           "--shell-escape" (TeXLive) aufgerufen werden!
% Gebrauch: -
% Optionen: update: nur konvertieren, wenn die EPS-Datei sich verändert hat
%           append/prepend: wenn _keine_ Endung angegeben wurde, wird hiermit
%           die Reihenfolge der Endungen bestimmt, nach der gesucht werden soll
% vor/nach: -
% Referenz: $miktex$/doc/latex/oberdiek/epstopdf.pdf
%\usepackage{epstopdf}


% >> rotating
% Funktion: Drehen von Grafiken/Tabellen
% Hinweise: -
% Gebrauch: sideways/sidewaysfigure
% Optionen: -
% vor/nach: -
% Referenz: $miktex$/doc/latex/rotating/rotating.pdf
%           rotating_examples.pdf
\usepackage[figuresright]{rotating}

% >> pgfplots
% Funktion: Erstellen von Plots
% Hinweise: -
% Gebrauch: -
% Optionen: -
% vor/nach: -
% Referenz: CTAN
%\usepackage{tikz-3dplot}
\usepackage{pgfplots}


%------------------------------------------------------------------------------%
%---- Bild-/Tabellenunterschriften --------------------------------------------%
%------------------------------------------------------------------------------%

% >> subfig
% Funktion: Erlaubt Unternummerierung von Bildern (1a, 1b, ...)
% Hinweise: -
% Gebrauch: \subfloat[caption]{figure}
% Optionen: -
% vor/nach: -
% Referenz: $miktex$/doc/latex/subfig/subfig.dvi
\usepackage{subfig}

% >> caption
% Funktion: Bild-/Tabellenunterschriften formatieren
% Hinweise: mindestens Version 2007/04/09 sonst Probleme mit hypcap
% Gebrauch: -
% Optionen: -
% vor/nach: -
% Referenz: $miktex$/doc/latex/caption/caption-deu.pdf
\usepackage{caption}

\captionsetup{                  % globale Option für caption und subcaption
  font=sf,		        % Schrift der Caption (Label+Text)
  format=hang,                  % Formatierung der Caption
  justification=RaggedRight,    % linksbündig bei mehreren caption-Zeilen
  singlelinecheck=true,         % true: einzelne Linie zentriert!
  labelfont=bf,			% Schrift des Labels
  textfont=rm,                  % Schrift des Textes
  position=bottom               % Normale Caption unter dem Float
}

% Wird mit dem KOMA-Script nicht mehr benötigt
%\captionsetup[table]{   % caption Optionen für Tabellen (caption über Tabelle)
%  position=top          % Vertauschen von above-/belowcaptionskip
%}

% Eigenes Labelformat definieren, vgl. caption-deu.pdf
\DeclareCaptionLabelFormat{Klammer}{#2)}

% Font settings wie sie später in der (Sub-)Caption verwendet werden:
% {\font {\labelfont #1#2}{\textfont #3}
% #1: <label>
% #2: <separator>
% #3: <text>
\captionsetup[subfloat]{    % subcaption Optionen
  font=small,               % Schrift der Caption (Label+Text)
  labelfont=rm,             % Schrift des Labels
  textfont=rm,              % Schrift des Textes
  captionskip=10pt,         % Abstand Caption <> Subfloat
  nearskip=0pt,             % Abstand Caption <> Folgetext (wenn Caption unten)
  labelformat=Klammer       % Eigenes Formt, nur eine Klammer: a)
}



%------------------------------------------------------------------------------%
%---- PDF-bezogene Pakete -----------------------------------------------------%
%------------------------------------------------------------------------------%

% >> Linkfarben
% Funktion: Linkfarben zur Verwendung im hyperref-Paket
% Hinweise: Die Digitalversion verwendet blau, grün, rot als Linkfarben
% Gebrauch: -
% Optionen: -
% vor/nach: vor hyperref, nach xcolor
% Referenz: $miktex$/doc/latex/graphics/grfguide.pdf
\definecolor{seclinks}{rgb}{0,0,0.5}  % blau, Verweise auf Abschnitte, Floats
\definecolor{citelinks}{rgb}{0,0.5,0} % grün, Verweise auf Quellen/Literatur
\definecolor{urllinks}{rgb}{0,0,0}    % schwarz, Verweise auf externe URLs


% >> hyperref
% Funktion: interne und externe Links für PDFs
% Hinweise: backref/pagebackref funktioniert nur bei Aufruf in usepackage
% Gebrauch: -
% Optionen: -
% vor/nach: Farbdefinition der Links
% Referenz: $miktex$/doc/latex/hyperref/manual.pdf
%           $miktex$/doc/latex/hyperref/paper.pdf
\usepackage{hyperref}

\hypersetup{%
  %hyperfootnotes=true,          % footnote verlinken zum footnotetext
  plainpages=false,             % true > Erzwingt arabische (1,2,...) Zahlen
  linktocpage=false,            % Seitenzahlen im Inhaltsverz. verlinkt
  unicode=true,                 % Unicode-Zeichen in Bookmarks, s.u.
  breaklinks=true               % Umbrüche in Links erlaubt
}

\ifprint % Druckversion
  \hypersetup{%
    colorlinks=false,            % Links werden umrandet dargestellt
    pdfborder=0 0 0,             % Umrandung wird deaktiviert
    pdfpagelayout=TwoColumnRight % Doppelseite, ungerade Seiten rechts
  }
\else % Digitalversion
  \hypersetup{%
    colorlinks=true,             % Links werden farblich dargestellt
    linkcolor=seclinks,          % "normal internal links" > Farbdefinition s.o.
    citecolor=citelinks,         % "bibliographical cites" > Farbdefinition s.o.
    urlcolor=urllinks,           % URLs > Farbdefinition s.o.
    pdfpagelayout=OneColumn      % eine Seite, kontinuierliches Scrollen
  }
\fi


% >> url
% Funktion: URLs definieren
% Hinweise: Erzeugt Verlinkung und Schriftartwechsel (Typewriter)
% Gebrauch: \url{}
% Optionen: -
% vor/nach: -
% Referenz: $miktex$/tex/latex/ltxmisc/url.sty
\usepackage{url}


% >> bookmark
% Funktion: Erweiterte Bookmark Verwaltung
% Hinweise: -
% Gebrauch: -
% Optionen: -
% vor/nach: -
% Referenz: $miktex$/doc/latex/oberdiek/bookmark.pdf
\usepackage{bookmark}

\bookmarksetup{%
  open=true,        % Bookmark-Baumstruktur ausgeklappt
  openlevel=1,      % Anzahl der Ebenen, die ausgeklappt werden
  numbered=true     % Nummerierung d. Bookmarks mit Nummern
}


% >> hypcap
% Funktion: Sprungmarke an den Anfang von Bilder/Tabellen verschieben
% Hinweise: Ursprünglich wird auf die caption gesprungen, so dass Bilder
%           etc. beim Klick auf Links zunächst nicht zu sehen ist.
%           Lösung: hypcap-package definiert die float-Umgebungen neu, so dass
%           standardmäßig an den Anfang gesprungen wird oder dorthin wo
%           \capstart manuell notiert wird
%
%           ACHTUNG: Bei Verwendung von hypcap und subfigure Probleme mit
%           Nummerierung,
%           hypcap v1.7 2007-04-11 mit caption v3.0l 2007-02-20 > counter-Fehler
%           hypcap v1.7 2007-04-11 funktioniert ab caption v3.0p 2007-04-09!
% Gebrauch: -
% Optionen: -
% vor/nach: -
% Referenz: -
\usepackage[all]{hypcap}
\renewcommand{\hypcapspace}{\baselineskip} % Zusätzliche Verschiebung nach oben

% If there are other float environments, that should automatically execute
% \capstart, then a redefinition with \hypcapredef can be tried
%\hypcapredef{sidewaysfigure}


% >> pdfpages
% Funktion: Externe PDF-Seiten hinzufügen
% Hinweise: -
% Gebrauch: \includepdf[key=val]{filename}
% Optionen: -
% vor/nach: -
% Referenz: $miktex$/doc/latex/pdfpages/pdfpages.pdf
\usepackage{pdfpages}



%------------------------------------------------------------------------------%
%---- Diverse Einstellungen und Pakete ----------------------------------------%
%------------------------------------------------------------------------------%

% >> listings
% Funktion: Quelltext-Formatierungen
% Hinweise: benötigt für Option "upquote=true" das textcomp Paket
% Gebrauch: -
% Optionen: -
% vor/nach: nach textcomp
% Referenz: $miktex$/doc/latex/listings/listings.pdf
\usepackage{listings}

% >> mdframed
% Funktion: Rahmen um Boxen
% Hinweise: -
% Gebrauch: -
% Optionen: -
% vor/nach: -
% Referenz: $miktex$/doc/latex/mdframed/mdframed.pdf
%\usepackage[framemethod=TikZ]{mdframed}


% >> ifthen
% Funktion: if-then-else Abfragen ermöglichen
% Hinweise: -
% Gebrauch: \ifthenelse{}{}{}
% Optionen: -
% vor/nach: -
% Referenz: $miktex$/doc/latex/base/ifthen.dvi
\usepackage{ifthen}


% >> varwidth
% Funktion: Boxen variabler Breite
% Hinweise: wie minipage, jedoch passt sich die Breite automatisch
%           der natürlichen Breite an
% Gebrauch: \begin{varwidth}{\linewidth}...\end{varwidth}
% Optionen: -
% vor/nach: -
% Referenz: $miktex$/doc/faq/english/html/FAQ-varwidth.html
\usepackage{varwidth}


% >> xspace
% Funktion: fügt Leerzeichen ein, wenn notwendig
% Hinweise: -
% Gebrauch: \xspace
% Optionen: -
% vor/nach: -
% Referenz: $miktex$/doc/latex/tools/xspace.dvi
\usepackage{xspace}


% >> \numberwithin
% Funktion: Nummerierung abhängig von Chapter/Section machen
% Hinweise: -
% Gebrauch: -
% Optionen: -
% vor/nach: nach amsmath, Befehl wird dort definiert
%           nach hyperref, damit die Links korrekt funktionieren
% Referenz: $miktex$/doc/latex/amsmath/amsldoc.dvi
\numberwithin{equation}{section}
\numberwithin{figure}{section}
\numberwithin{table}{section}



%------------------------------------------------------------------------------%
%---- Überprüfung von Satzspiegel und Layout ----------------------------------%
%------------------------------------------------------------------------------%

% Hinweis: alle Pakete dienen nur der Überprüfung vom Satzspiegeleinstellungnen
% und sollten aus Performancegründen später deaktiviert werden!

% >> layout
% Funktion: Grafische Ausgabe des Satzspiegel (skaliert)
% Hinweise: -
% Gebrauch: \layout im Text platzieren
% Optionen: -
% vor/nach: nach der Satzspiegelberechnung
% Referenz: $miktex$/doc/latex/tools/layout.dvi
%\usepackage{layout}


% >> layouts
% Funktion: Anzeige des Satzspiegel und Parametern
% Hinweise: -
% Gebrauch: \printinunitsof{mm} (auf Millimeter-Angaben umschalten)
%           \pagevalues (Zahlenwerte der Längen)
%           \pagediagram (Diagramm des Layouts)
% Optionen: -
% vor/nach: nach der Satzspiegelberechnung
% Referenz: $miktex$/doc/latex/layouts/layman.pdf
%\usepackage{layouts}
%\printinunitsof{mm}


% >> showframe
% Funktion: Zeichnet Rahmen für Textkörper u. Randtext in den Seitenhintergrund
% Hinweise: Using the package option "noframe" you can draw the frames later only 
%           for a single page with \AddToShipoutPicture*{\ShowFramePicture}. Other 
%           useful options maybe eso-pic's "colorgrid" or "grid".
% Gebrauch: -
% Optionen: colorgrid: farbige Gitterlinien
%           texcoord: linke obere Ecke > Nullpunkt
%           gridunit: Einheit für das Gitter wählen
% vor/nach: nach der Satzspiegelberechnung
% Referenz: $miktex$/doc/latex/eso-pic/eso-pic.dvi
%\usepackage[colorgrid,texcoord,gridunit=mm]{showframe}
%\usepackage[noframe]{showframe}
%\usepackage{showframe}


% >> blindtext
% Funktion: Erzeugt Blindtext, nur für Testzwecke
% Hinweise: sprachenabhängig nach babel-Einstellung
% Gebrauch: \blindtext oder \blindtext[Anzahl]
% Optionen: -
% vor/nach: babel Paket
% Referenz: $miktex$\doc\latex\blindtext\blindtext.pdf
\usepackage{blindtext}

%------------------------------------------------------------------------------%
%---- Literaturverwaltung -----------------------------------------------------%
%------------------------------------------------------------------------------%

% >> csquotes
% Funktion: -
% Hinweise: -
% Gebrauch: -
% Optionen: -
% vor/nach: babel Paket
% Referenz: -
\usepackage[babel,german=quotes]{csquotes}

% >> biblatex
% Funktion: -
% Hinweise: -
% Gebrauch: -
% Optionen: -
% vor/nach: babel Paket
% Referenz: -
\usepackage[
		backend=biber,
		style=authoryear,
%		bibstyle=authoryear,
		citestyle=authoryear
%		natbib=
%		mcite=,
		]{biblatex}
		
%------------------------------------------------------------------------------%
%---- Weitere Einstellungen und Pakete ----------------------------------------%
%------------------------------------------------------------------------------%	
% >> ifdraft
% Funktion: Überprüft ob draft modus an ist
% Hinweise: -
% Gebrauch: \ifdraft{<draft case>}{<other case>}
% Optionen: -
% vor/nach: -
% Referenz: -
\usepackage{ifdraft}

% >> todonotes
% Funktion: Ermöglicht
% Hinweise: -
% Gebrauch: \todo{...}, \missingfigure{...}, \listoftodos
% Optionen: -
% vor/nach: -
% Referenz: -
\usepackage[colorinlistoftodos,prependcaption,textsize=tiny]{todonotes}