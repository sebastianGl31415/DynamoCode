%------------------------------------------------------------------------------%
%------------------------------------------------------------------------------%
%--- Titel:    ...                                                          ---%
%--- Autor:    ...                                                          ---%
%--- Vorlage:  Dipl.-Ing. Timo Pe <timo.pe@tu-harburg.de>                   ---%
%---           Institut für Flugzeug-Systemtechnik                          ---%
%---           Neßpriel 5, 21129 Hamburg                                    ---%
%---           http://www.fst.tu-harburg.de/                                ---%
%------------------------------------------------------------------------------%
%------------------------------------------------------------------------------%

% Hier werden die Kommandos initialisiert, die im Dokument mehrfach benötigt
% werden. Da sie eher dokumentenspezifisch sind, befinden sie sich nicht in
% der Präambel.

%------------------------------------------------------------------------------%
%---- Farbdefinition ----------------------------------------------------------%
%------------------------------------------------------------------------------%

\definecolor{backgroundgrey}{gray}{0.5}
\definecolor{lightgray}{gray}{0.5}
\definecolor{mediangray}{gray}{0.3125}
\definecolor{stronggray}{gray}{0.25}

\definecolor{box}{gray}{0.5}
\definecolor{keywords}{RGB}{0,0,255}
\definecolor{comments}{gray}{0.5}
\definecolor{strings}{RGB}{160,32,240}


%------------------------------------------------------------------------------%
%---- Abkürzungen/Kurzschreibweisen -------------------------------------------%
%------------------------------------------------------------------------------%

% Abkürzungen: kleinerer Abstand zwischen den Buchstaben, der zudem nicht
% umbrochen werden kann
\newcommand*\zB{z.\,B.\xspace}
\newcommand*\dah{d.\,h.\xspace} % Achtung \dh ist leider schon vergeben!
\newcommand*\ua{u.\,a.\xspace}
\newcommand*\uU{u.\,U.\xspace}
\newcommand*\idR{i.\,d.\,R.\xspace}
\newcommand*\iA{i.\,A.\xspace}



%------------------------------------------------------------------------------%
%---- Typografisches, Text ----------------------------------------------------%
%------------------------------------------------------------------------------%

% >> \textsubscript
% Funktion: Tiefgestelltes Zeichen im Text (Pendant zu \textsuperscript)
% Hinweise: Die KOMA-Script-Klassen definieren dieses Kommando bereits,
%           daher unnötig bei Verwendung dieser Klassen
% Gebrauch: \textsubscript{}
% Optionen: -
% vor/nach: -
% Referenz: DE-TeX-FAQ, Abschnitt 8.5.17, http://www.dante.de/faq/de-tex-faq/
% \makeatletter
% \DeclareRobustCommand*\textsubscript[1]{%
%  \@textsubscript{\selectfont#1}}
% \newcommand{\@textsubscript}[1]{%
%  {\m@th\ensuremath{_{\mbox{\fontsize\sf@size\z@#1}}}}}
% \makeatother


% >> \dequote \frquote \enquote
% Funktion: deutsche/französische/englische Anführungszeichen
% Hinweise: -
% Gebrauch: \dequote{...} \frquote{...} \enquote{...}
% Optionen: -
% vor/nach: -
% Referenz: $groups$/de.comp.text.tex/browse_thread/thread/283337509f97cd3f
\newcommand{\dequote}[1]{\glqq#1\grqq}
\newcommand{\frquote}[1]{\guillemotright#1\guillemotleft} % fontenc-Paket
%\newcommand{\enquote}[1]{\textquotedblleft#1\textquotedblright}



%------------------------------------------------------------------------------%
%---- Listings / frames -------------------------------------------------------%
%------------------------------------------------------------------------------%

%\lstset{
%	language=Python,
%	numberbychapter=true,
%	captionpos=t,
%	basicstyle=\footnotesize\ttfamily,	% Standardschrift
%	numberstyle=\tiny,					% Stil der Zeilennummern
%	numbers=left,
%	stepnumber=1,						% Abstand zwischen den Zeilennummern
%	numbersep=5pt,						% Abstand der Nummern zum Text
%	tabsize=2,							% Groesse von Tabs
%	extendedchars=true,         			%
%	breaklines=true,            			% Zeilen werden Umgebrochen
%	keywordstyle=\color{keywords},
%    	commentstyle=\color{comments},
%	stringstyle=\color{strings},
%    	frame=none,
%	title=\lstname,
%	showspaces=false,           % Leerzeichen anzeigen
%	showtabs=false,             % Tabs anzeigen
%	linewidth=\textwidth,
%	xleftmargin=\bigskipamount,
%	showstringspaces=false      % Leerzeichen in Strings anzeigen ?
% }
%\renewcommand{\lstlistingname}{\textsc{FEniCS}-Programm}
%\renewcommand{\lstlistlistingname}{Verzeichnis der \textsc{FEniCS}-Programme}
%\DeclareCaptionFormat{listingbox}{\colorbox{box}{\parbox{\textwidth}{\quad #1#2#3}}}
%\DeclareCaptionFont{listingtitle}{\color{white}\sffamily\textbf}
%\DeclareCaptionFont{listingname}{\color{white}}
%\captionsetup[lstlisting]{
%	format=listingbox,
%	labelfont=listingtitle,
%	singlelinecheck=true,
%	font=listingname
%	}
%

%------------------------------------------------------------------------------%
%---- Frames ------------------------------------------------------------------%
%------------------------------------------------------------------------------%


% >> literaturempfehlung
% Funktion: Hervorhebung für Literaturempfehlungen
% Hinweise: basiert auf mdframed-Paket
% Gebrauch: -
% Optionen: -
% vor/nach: -
% Referenz: $miktex$/doc/latex/mdframed/mdframed.pdf
%\newenvironment{literaturempfehlung}[1][Literaturempfehlung]%
%{%
%	\begin{minipage}{\linewidth}% verhindert Seitenumbruch
%	\setlength{\parskip}{5pt plus2pt}% Abstände zwischen Absätzen
%	\vspace{2pt}%
%	\hspace{1em}\textbf{#1}\smallskip
%    % mdframed-Paket: umrahmte Boxen
%	\begin{mdframed}[innerlinewidth=2pt, linecolor=lightgray, bottomline=false, rightline=false, skipbelow=4pt]%
%}%
%{%
%    \end{mdframed}\end{minipage}%
%}%

\newtheorem{problem}{Problem}
%------------------------------------------------------------------------------%
%---- Quellenangaben ----------------------------------------------------------%
%------------------------------------------------------------------------------%

% \autoref{[equationlabel]} ergibt z.B.: "Gleichung 2.4"
% Zahl soll aber in Klammern stehen "Gleichung (2.4)":
\newcommand{\eqnautoref}[1]{%
\hyperref[#1]{\equationautorefname{}~(\ref*{#1})}%
}%

% Flexiblere Links auf Gleichungen:
% [#1]: Beliebiger Text z.B. "Bedingung", wenn nicht verwendet wird nur die
%       Gleichungsnummer in Klammern verlinkt
% #2: equationlabel
%
\newcommand{\eqnref}[2][]{%
\hyperref[#2]{#1 (\ref*{#2})}%
}%

% \autoref{[appendix_section]} ergibt z.B.: "Abschnitt A.1", soll aber
% "Anhang A.1" ergeben. Geht scheinbar nur manuell über selbstdefiniertes
% Kommando:
\newcommand{\apxautoref}[1]{%
\hyperref[#1]{\appendixautorefname{} \ref*{#1}}%
}%




%------------------------------------------------------------------------------%
%---- Beschreibungen/Erläuterungen mit Überschriftszeile ----------------------%
%------------------------------------------------------------------------------%

% Umgebung für Erläuterungen/Beschreibungen etc.

% Syntax-Beispiel:
% \begin{itemheadline} % siehe macros.tex
%   \headline{rot:}
%     Die Farbe der Liebe
%   \headline[\bfseries]{grün:}
%     Die Farbe der Hoffnung
% \end{itemheadline}

\newenvironment{itemheadline}%
{%
    \bgroup%
    %\setdefaultleftmargin{5.0ex}{2.5ex}{2.5ex}{2.5ex}{2.5ex}{2.5ex}%
    \setlength{\plparsep}{0.8ex}%
    \setlength{\plitemsep}{1.2ex}%
    \setlength{\labelsep}{\labelsep+0.4ex}% calc-Paket
    \begin{compactitem}%
}%
{%
    \end{compactitem}%
    \egroup%
}%

% nummerierte Beschreibung
% Die "label" (= Zahlen) sind etwas breiter als der "bullet", zum optischen
% Ausgleich wird der Abstand des labels vom item im Vergleich zur Auflistung
% erniedrigt

% [#1]: Label, z.B. "1." oder "A)" oder "I.)" etc.

% Syntax-Beispiel:
% \begin{enumheadline}[a)] % siehe macros.tex
%   \headline{rot:}
%     Die Farbe der Liebe
%   \headline[\bfseries]{grün:}
%     Die Farbe der Hoffnung
% \end{enumheadline}

\newenvironment{enumheadline}[1][1.]%
{%
    \bgroup%
    %\setdefaultleftmargin{5.0ex}{2.5ex}{2.5ex}{2.5ex}{2.5ex}{2.5ex}%
    \setlength{\plparsep}{0.8ex}%
    \setlength{\plitemsep}{1.2ex}%
    \begin{compactenum}[#1]%
}%
{%
    \end{compactenum}%
    \egroup%
}%

% headline
% \par erzeugt einen neuen Absatz, \nopagebreak verhindert, dass der Name der
% Teilfunktion von der Erklärung durch einen Seitenumbruch getrennt werden.
% #1: Schrifteinstellungen
% #2: Text
\newcommand{\headline}[2][\itshape]{%
    \item {#1#2}\par\nopagebreak%
}%


%------------------------------------------------------------------------------%
%---- Mathematische Makros ----------------------------------------------------%
%------------------------------------------------------------------------------%

\newcommand{\p}{\partial}
\newcommand{\dd}[2]{\frac{\mathrm{d} #1}{\mathrm{d} #2}}
\newcommand{\ddsqr}[2]{\frac{\mathrm{d}^2 #1}{\mathrm{d} #2^2}}
\newcommand{\pd}[2]{\frac{\p #1}{\p #2}}
\newcommand{\ppd}[3]{\frac{\p^2 #1}{\p #2\p #3}}
\newcommand{\pppd}[4]{\frac{\p^3 #1}{\p #2\p #3\p #4}}
\newcommand{\pdsqr}[2]{\frac{\p^2 #1}{\p #2^2}}
\newcommand{\pdsqrr}[3]{\frac{\p^2 #1}{\p #2 \p #3}}
%------------------------------------------------------------------------------%
\newcommand{\z}[1]{\ensuremath{\stackrel[(z)]{}{#1}}\!\vphantom{#1}}
\newcommand{\x}[1]{\ensuremath{\stackrel[(x)]{}{#1}}\!\vphantom{#1}}
\newcommand{\phy}[1]{\ensuremath{{\left\langle #1 \right\rangle}}}
\newcommand{\christ}[2]{\ensuremath{\mathit{\Gamma}^{#1}_{#2}}}
%------------------------------------------------------------------------------%
\newcommand{\jump}[1]{\ensuremath{\left[\!\left[#1\right]\!\right]}}
%------------------------------------------------------------------------------%
\newcommand{\psiV}{\ensuremath{\stackrel[V]{}{\psi}\!\vphantom{\psi}}}
\newcommand{\psiA}{\ensuremath{\stackrel[A]{}{\psi}\!\vphantom{\psi}}}
\newcommand{\psiL}{\ensuremath{\stackrel[L]{}{\psi}\!\vphantom{\psi}}}
\newcommand{\bmphiV}{\ensuremath{\stackrel[V]{}{\bm{\psi}}\!\vphantom{\bm{\psi}}}}
\newcommand{\bmpsiA}{\ensuremath{\stackrel[A]{}{\bm{\psi}}\!\vphantom{\bm{\psi}}}}
\newcommand{\bmpsiL}{\ensuremath{\stackrel[L]{}{\bm{\psi}}\!\vphantom{\bm{\psi}}}}
\newcommand{\phiA}{\ensuremath{\stackrel[A]{}{\phi}\!\vphantom{\phi}}}
\newcommand{\phiL}{\ensuremath{\stackrel[L]{}{\phi}\!\vphantom{\phi}}}
\newcommand{\bmphiA}{\ensuremath{\stackrel[A]{}{\bm{\phi}}\!\vphantom{\bm{\phi}}}}
\newcommand{\bmphiL}{\ensuremath{\stackrel[L]{}{\bm{\phi}}\!\vphantom{\bm{\phi}}}}
\newcommand{\pV}{\ensuremath{\stackrel[V]{}{p}\!\vphantom{p}}}
\newcommand{\pA}{\ensuremath{\stackrel[A]{}{p}\!\vphantom{p}}}
\newcommand{\pL}{\ensuremath{\stackrel[L]{}{p}\!\vphantom{p}}}
\newcommand{\bmpV}{\ensuremath{\stackrel[V]{}{\bm{p}}\!\vphantom{\bm{p}}}}
\newcommand{\bmpA}{\ensuremath{\stackrel[A]{}{\bm{p}}\!\vphantom{\bm{p}}}}
\newcommand{\bmpL}{\ensuremath{\stackrel[L]{}{\bm{p}}\!\vphantom{\bm{p}}}}
\newcommand{\sV}{\ensuremath{\stackrel[V]{}{s}\!\vphantom{s}}}
\newcommand{\sA}{\ensuremath{\stackrel[A]{}{s}\!\vphantom{s}}}
\newcommand{\sL}{\ensuremath{\stackrel[L]{}{s}\!\vphantom{s}}}
\newcommand{\bmsV}{\ensuremath{\stackrel[V]{}{\bm{s}}\!\vphantom{\bm{s}}}}
\newcommand{\bmsA}{\ensuremath{\stackrel[A]{}{\bm{s}}\!\vphantom{\bm{s}}}}
\newcommand{\bmsL}{\ensuremath{\stackrel[L]{}{\bm{s}}\!\vphantom{\bm{s}}}}
%------------------------------------------------------------------------------%
\newcommand{\Dfrak}{\ensuremath{\mathfrak{D}}}
\newcommand{\bmDfrak}{\ensuremath{\bm{\mathfrak{D}}}}
\newcommand{\efrak}{\ensuremath{\mathfrak{e}}}
\newcommand{\bmefrak}{\ensuremath{\bm{\mathfrak{e}}}}
\newcommand{\Efrak}{\ensuremath{\mathfrak{E}}}
\newcommand{\bmEfrak}{\ensuremath{\bm{\mathfrak{E}}}}
\newcommand{\Hfrak}{\ensuremath{\mathfrak{H}}}
\newcommand{\bmHfrak}{\ensuremath{\bm{\mathfrak{H}}}}
%------------------------------------------------------------------------------%
\newcommand{\qL}{\ensuremath{\stackrel[L]{}{q}\!\vphantom{q}}}
\newcommand{\qA}{\ensuremath{\stackrel[A]{}{q}\!\vphantom{q}}}
\newcommand{\qV}{\ensuremath{\stackrel[V]{}{q}\!\vphantom{q}}}
%------------------------------------------------------------------------------%
\newcommand{\qf}{\ensuremath{q^\text{f}}}
\newcommand{\qfL}{\ensuremath{\stackrel[L]{}{q}^\text{f}\!\vphantom{\qf}}}
\newcommand{\qfA}{\ensuremath{\stackrel[A]{}{q}^\text{f}\!\vphantom{\qf}}}
\newcommand{\qfV}{\ensuremath{\stackrel[V]{}{q}^\text{f}\!\vphantom{\qf}}}
%------------------------------------------------------------------------------%
\newcommand{\qp}{\ensuremath{q^\text{p}}}
\newcommand{\qpL}{\ensuremath{\stackrel[L]{}{q}^\text{p}\!\vphantom{\qp}}}
\newcommand{\qpA}{\ensuremath{\stackrel[A]{}{q}^\text{p}\!\vphantom{\qp}}}
\newcommand{\qpV}{\ensuremath{\stackrel[V]{}{q}^\text{p}\!\vphantom{\qp}}}
%------------------------------------------------------------------------------%
\newcommand{\jf}{\ensuremath{j^\text{f}}}
\newcommand{\jfA}{\ensuremath{\stackrel[A]{}{j}^\text{f}\!\vphantom{\jf}}}
\newcommand{\jfL}{\ensuremath{\stackrel[L]{}{j}^\text{f}\!\vphantom{\jf}}}
\newcommand{\bmjf}{\ensuremath{\bm{j}^\text{f}}}
\newcommand{\bmjfL}{\ensuremath{\stackrel[L]{}{\bmj}^\text{f}\!\vphantom{\bmjf}}}
\newcommand{\bmjfA}{\ensuremath{\stackrel[A]{}{\bmj}^\text{f}\!\vphantom{\bmjf}}}
%------------------------------------------------------------------------------%
\newcommand{\jp}{\ensuremath{j^\text{p}}}
\newcommand{\jpL}{\ensuremath{\stackrel[L]{}{j}^\text{p}\!\vphantom{\jp}}}
\newcommand{\jpA}{\ensuremath{\stackrel[A]{}{j}^\text{p}\!\vphantom{\jp}}}
\newcommand{\bmjp}{\ensuremath{\bm{j}^\text{p}}}
\newcommand{\bmjpL}{\ensuremath{\stackrel[L]{}{\bmj}^\text{p}\!\vphantom{\bmjp}}}
\newcommand{\bmjpA}{\ensuremath{\stackrel[A]{}{\bmj}^\text{p}\!\vphantom{\bmjp}}}
%------------------------------------------------------------------------------%
\newcommand{\bmj}{\ensuremath{\bm{j}}}
\newcommand{\bmjL}{\ensuremath{\stackrel[L]{}{\bmj}\!\vphantom{\bmj}}}
\newcommand{\bmjA}{\ensuremath{\stackrel[A]{}{\bmj}\!\vphantom{\bmj}}}
%------------------------------------------------------------------------------%
\newcommand{\Bref}{\ensuremath{B_\bez}}
\newcommand{\Eref}{\ensuremath{E_\bez}}
\newcommand{\lref}{\ensuremath{l_\bez}}
\newcommand{\tref}{\ensuremath{t_\bez}}
\newcommand{\vref}{\ensuremath{v_\bez}}
%------------------------------------------------------------------------------%
\newcommand{\BL}{\ensuremath{\stackrel[L]{}{B}\!\vphantom{B}}}
\newcommand{\BA}{\ensuremath{\stackrel[A]{}{B}\!\vphantom{B}}}
\newcommand{\bmBL}{\ensuremath{\stackrel[L]{}{\bm{B}}\!\vphantom{\bm{B}}}}
\newcommand{\bmBA}{\ensuremath{\stackrel[A]{}{\bm{B}}\!\vphantom{\bm{B}}}}
%------------------------------------------------------------------------------%
\newcommand{\rhoL}{\ensuremath{\stackrel[L]{}{\rho}\!\vphantom{\rho}}}
\newcommand{\rhoA}{\ensuremath{\stackrel[A]{}{\rho}\!\vphantom{\rho}}}
\newcommand{\rhoV}{\ensuremath{\stackrel[V]{}{\rho}\!\vphantom{\rho}}}
%------------------------------------------------------------------------------%
\newcommand{\er}{\ensuremath{\bm{e}_r}}
\newcommand{\etheta}{\ensuremath{\bm{e}_\theta}}
\newcommand{\ephi}{\ensuremath{\bm{e}_\vp}}
%------------------------------------------------------------------------------%
\newcommand{\nosum}[1]{\ensuremath{{\underline{#1}}}}
%------------------------------------------------------------------------------%
\newcommand{\eps}{\ensuremath{\epsilon}}
\newcommand{\0}{\ensuremath{\emptyset}}
\newcommand{\const}{\ensuremath{\mathrm{const.}}}
\newcommand{\bez}{\ensuremath{\text{ref.}}}
\newcommand{\Rey}{\ensuremath{\mathit{Re}}}
\newcommand{\Remag}{\ensuremath{\Rey_\text{mag.}}}
%------------------------------------------------------------------------------%
%\newcommand{\inv}[1]{\ensuremath{\stackrel[]{-1}{#1}}}
\newcommand{\inv}[1]{\stackrel{-1}{#1}\!\!\vphantom{#1}}
%------------------------------------------------------------------------------%
\newcommand{\tripleF}[3]{\ensuremath{F_{#1 1}F_{#2 2}F_{#3 3}}}
%------------------------------------------------------------------------------%
\renewcommand{\d}[1]{\ensuremath{\mathrm{d}#1} }
\newcommand{\drm}[1]{\ensuremath{\,\mathrm{d}#1}}
\newcommand{\irm}{\ensuremath{\mathrm{i}}}
%------------------------------------------------------------------------------%
\newcommand{\epsijk}{\ensuremath{\epsilon_{ijk}}}
\newcommand{\grad}[1]{\operatorname{grad}\left(#1\right)}
\newcommand{\rot}[2][]{\operatorname{rot}_{#1}\left(#2\right)}
\renewcommand{\div}[1]{\ensuremath{\operatorname{div}\left(#1\right)}}
\newcommand{\laplace}{\ensuremath{\Delta}}
\newcommand{\deter}[1]{\det{\left(\bm{#1}\right)}}
\newcommand{\spur}[1]{\operatorname{spur}\left(\bm{#1}\right)}
\newcommand{\LOperator}[2][]{
	\ifthenelse{\equal{#1}{}}
		{\ensuremath{L\!\left[#2\right]}}
		{\ensuremath{L_{#1}\!\left[#2\right]}}
		}
\newcommand{\DOperator}[3][]{
	\ifthenelse{\equal{#1}{}}
		{	\ifthenelse{\equal{#2}{}}
			{\ensuremath{D\!\left[#3\right]}}
			{\ensuremath{D^{\left(#2\right)}\!\left[#3\right]}}}
		{	\ifthenelse{\equal{#2}{}}
			{\ensuremath{D_{#1}\!\left[#3\right]}}
			{\ensuremath{D_{#1}^{\left(#2\right)}\!\left[#3\right]}}
		}
	}
\newcommand{\DThetaPhiOperator}[2]{
	\ensuremath{
		\mathcal{D}_{\theta,\vp}\!\left[#1,#2\right]
		}
	}
\newcommand{\bmDThetaPhiOperator}[1]{
	\ensuremath{
		\bm{\mathcal{D}}_{\theta,\vp}\!\left[#1\right]
		}
	}
\newcommand{\bmDThetaPhi}{\ensuremath{\bm{\mathcal{D}}_{\theta,\vp}}}
\newcommand{\landau}[1]{\ensuremath{\mathcal{O}\left(#1\right)}}
\newcommand{\order}[1]{\ensuremath{\mathcal{O}\left(#1\right)}}
\newcommand{\levicivi}[1]{\ensuremath{\varepsilon_{#1}}}
\newcommand{\abs}[1]{\ensuremath{\left| #1 \right|}}
\newcommand{\norm}[1]{\ensuremath{\left|\!\left| #1 \right|\!\right|}}
\newcommand{\scalar}[2]{\ensuremath{\left\langle #1,#2\right\rangle}}
\newcommand{\conj}[1]{\ensuremath{\overline{#1}}}
%------------------------------------------------------------------------------%
\newcommand{\vr}{\varrho}
\newcommand{\vt}{\vartheta}
\newcommand{\vp}{\varphi}
%------------------------------------------------------------------------------%
\newcommand{\si}[1]{\sin{\left(#1\right)}}
\newcommand{\co}[1]{\cos{\left(#1\right)}}
\newcommand{\ta}[1]{\tan{\left(#1\right)}}
\newcommand{\cota}[1]{\cot{\left(#1\right)}}
\newcommand{\sih}[1]{\sinh{\left(#1\right)}}
\newcommand{\coh}[1]{\cosh{\left(#1\right)}}
\newcommand{\tah}[1]{\tanh{\left(#1\right)}}
\newcommand{\siq}[1]{\sin^2{\left(#1\right)}}
\newcommand{\coq}[1]{\cos^2{\left(#1\right)}}
\newcommand{\taq}[1]{\tan^2{\left(#1\right)}}
\newcommand{\asi}[1]{\arcsin{\left(#1\right)}}
\newcommand{\aco}[1]{\ensuremath{\arccos{\left(#1\right)}}}
\newcommand{\ata}[1]{\ensuremath{\arctan{\left(#1\right)}}}
%------------------------------------------------------------------------------%
\newcommand{\expo}[1]{\ensuremath{\exp{\left(#1\right)}}}
\newcommand{\alegend}[3][]{
	\ifthenelse{\equal{#1}{}}
		{\ensuremath{P_{#2}^{#3}}}
		{\ensuremath{P_{#2}^{#3}\!\left(#1\right)}}
		}
\newcommand{\dalegend}[3][]{
	\ifthenelse{\equal{#1}{}}
		{\ensuremath{{P_{#2}^{#3}}'}}
		{\ensuremath{{P_{#2}^{#3}}'\!\left(#1\right)}}
		}
\newcommand{\ddalegend}[3][]{
	\ifthenelse{\equal{#1}{}}
		{\ensuremath{{P_{#2}^{#3}}''}}
		{\ensuremath{{P_{#2}^{#3}}''\!\left(#1\right)}}
		}		
\newcommand{\Plm}[1]{\ensuremath{\alegend[#1]{l}{m}}}
\newcommand{\Plmcos}{\ensuremath{\alegend[\co{\theta}]{l}{m}}}
%------------------------------------------------------------------------------%
\newcommand{\pair}[2]{\ensuremath{
	\begin{Bmatrix}
	#1 \\ #2
	\end{Bmatrix}
	}}
%------------------------------------------------------------------------------%
\newcommand{\kreis}[1]{\ \unitlength1ex\begin{picture}(2.5,2.5)%
\put(0.75,0.75){\circle{2.5}}\put(0.75,0.75){\makebox(0,0){#1}}\end{picture}}
