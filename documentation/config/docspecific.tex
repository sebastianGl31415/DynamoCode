%------------------------------------------------------------------------------%
%------------------------------------------------------------------------------%
%--- Titel:    ...                                                          ---%
%--- Autor:    ...                                                          ---%
%--- Vorlage:  Dipl.-Ing. Timo Pe <timo.pe@tu-harburg.de>                   ---%
%---           Institut für Flugzeug-Systemtechnik                          ---%
%---           Neßpriel 5, 21129 Hamburg                                    ---%
%---           http://www.fst.tu-harburg.de/                                ---%
%------------------------------------------------------------------------------%
%------------------------------------------------------------------------------%


%------------------------------------------------------------------------------%
%---- PDF-Metadaten -----------------------------------------------------------%
%------------------------------------------------------------------------------%

% >> hypersetup
% Funktion: PDF-Metadaten festlegen
% Hinweise: -
% Gebrauch: -
% Optionen: -
% vor/nach: -
% Referenz: $miktex$/doc/latex/hyperref/manual.pdf
%           $miktex$/doc/latex/hyperref/paper.pdf
\hypersetup{
  pdftitle = {Autofrettage-Probleme in krummlinigen Koordinatensystemen},%
  pdfkeywords = {Autofrettage-Probleme, Plastizität,
  Zylinderkoordinaten, FEniCs},%
  pdfauthor = {Sebastian Glane, Laurent Bernier},%
  pdfcreator = {LaTeX, pdfTeX},  %
  pdfsubject = {Autofrettage-Probleme}%
}



%------------------------------------------------------------------------------%
%---- Symbole -----------------------------------------------------------------%
%------------------------------------------------------------------------------%

% >> Diverse mathematische Symbol-Pakete
% Funktion: Diverse Symbole werden zur Verfügung gestellt
% Hinweise: Eine vollständige Auflistung ist der "Comprehensive LaTeX Symbol
%           List" (symbols-a4.pdf) oder den einzelnen Paket-Beschreibung
% Gebrauch: -
% Optionen: -
% vor/nach: -
% Referenz: Comprehensive LaTeX Symbol List (symbols-a4.pdf)
%\usepackage{mathabx}   % mathtest.pdf/symbols-a4.pdf
%\usepackage{MnSymbol}  % Math Symbols $miktex$/doc/font/mnsymbol/MnSymbol.pdf
%\usepackage{amssymb}   % AMS-Symbole
%\usepackage{upgreek}   % griechische, aufrechte Symbole
%\usepackage{mathcomp}  %


% >> Diverse Symbol-Pakete (keine Mathesymbole!)
% Funktion: Diverse Symbole werden zur Verfügung gestellt
% Hinweise: Eine vollständige Auflistung ist der "Comprehensive LaTeX Symbol
%           List" (symbols-a4.pdf) oder den einzelnen Paket-Beschreibung
% Gebrauch: -
% Optionen: -
% vor/nach: -
% Referenz: Comprehensive LaTeX Symbol List (symbols-a4.pdf)
%\usepackage{textcomp}  % "Text-Mode" Symbole
%\usepackage{marvosym}  % siehe $miktex$/doc/fonts/marvosym/marvodoc.pdf
%\usepackage{pifont}    %
%\usepackage{wasysym}   %



%------------------------------------------------------------------------------%
%---- Tabellen ---------------------------------------------------------------%
%------------------------------------------------------------------------------%

% >> \newcolumntype
% Funktion: Eigene Spaltentypen definieren
% Hinweise: Rechtsbündige/zentrierte Spalte auf Basis von tabularx/ragged2e
%           \hspace{0pt} > damit Silbentrennung auch dann funktioniert,
%           wenn erstes Wort breiter als Spalte.
% Gebrauch: -
% Optionen: -
% vor/nach: nach array-Paket, nach tabularx-Paket
% Referenz: $miktex$/doc/latex/tools/array.dvi
%           $miktex$/doc/latex/tools/tabularx.dvi
%           tabsatz.pdf
%\newcolumntype{L}{>{\RaggedRight\arraybackslash\hspace{0pt}}X}
%\newcolumntype{R}{>{\RaggedLeft\arraybackslash\hspace{0pt}}X}
%\newcolumntype{C}{>{\Centering\arraybackslash\hspace{0pt}}X}


% >> \extrarowheight
% Funktion: Extra-Höhe in Tabellen und Arrays
% Hinweise: In der Standardeinstellung ist zu wenig Platz nach oben
% Gebrauch: -
% Optionen: -
% vor/nach: array-Paket benötigt
% Referenz: -
\setlength{\extrarowheight}{1.5pt}



%------------------------------------------------------------------------------%
%---- Konfiguration von Listen/Aufzählungen -----------------------------------%
%------------------------------------------------------------------------------%

% Erläuterung der Längen in "The LaTeX Companion", S. 145, 2nd. edition (2004)
% \topsep, \partopsep, \itemsep, \listparindent, \itemindent, \labelwidth,
% \labelsep, \parsep

% Formatierung der Listen/Aufzählungen
% Einrückung der Ebenen (immer alle 6 Ebenen angeben):
\setdefaultleftmargin{5.0ex}{2.5ex}{2.5ex}{2.5ex}{2.5ex}{2.5ex}

% Problem: \itemsep (für itemize) kann nicht global definiert werden
% > deshalb: compactitem
% Standard (itemize): 5.0pt plus 2.5pt minus 1.0pt
\setlength{\plitemsep}{2.0pt plus 1.0pt minus 0.5pt} 

% vertik. Abstand von Absätzen innerhalb eines items
\setlength{\plparsep}{3.0pt plus 1.0pt minus 0.5pt}


% >> Verkleinertes bullet-Symbol
% Funktion: Verkleinertes bullet-Symbol
% Hinweise: Durch ein einfaches \small falsche Ausrichtung, deshalb
%           Verschiebung mit raisebox
% Gebrauch: -
% Optionen: -
% vor/nach: nach calc (Längenberechnungen)
% Referenz: $groups$/de.comp.text.tex/browse_thread/thread/3c7a42c50b3c7267
\newcommand{\textbulletsmall}{%
  \raisebox{(\heightof{\textbullet}-\heightof{\small\textbullet})/2}%
  {\small\textbullet}%
}%


% Auflistungszeichen der itemize-Umgebung neu definieren
\renewcommand\labelitemi{\textbulletsmall}
\renewcommand\labelitemii{$\triangleright$}
\renewcommand\labelitemiii{$\diamond$}
\renewcommand\labelitemiv{--}



%------------------------------------------------------------------------------%
%---- Bild-/Tabellenunterschriften --------------------------------------------%
%------------------------------------------------------------------------------%

% >> \figurename\tablename
% Funktion: Umbenennung der \...name
% Hinweise: -
% Gebrauch: -
% Optionen: -
% vor/nach: ggf. nach babel
% Referenz: DE-TeX-FAQ, Abschnitt 8.5.9, http://www.dante.de/faq/de-tex-faq/
% Wenn Babel mit Option ngerman verwendet wird:
\addto\captionsngerman{%
 \renewcommand{\figurename}{Abbildung}    % Figure-Name
 \renewcommand{\tablename}{Tabelle}  % Table-Name
}%

% Nur wenn Babel nicht verwendet wird (sonst wird es wieder von Babel
% überschrieben:
%\renewcommand{\figurename}{Bild}    % Figure-Name
%\renewcommand{\tablename}{Tabelle}  % Table-Name



%------------------------------------------------------------------------------%
%---- Autobenennung bei Referenzen --------------------------------------------%
%------------------------------------------------------------------------------%

% >> \...autorefname
% Funktion: Namen für \autoref-Links setzen
% Hinweise: \autoref erzeugt automatisch einen Link für z.B. siehe "Bild 1.1",
%           die Umgebung wird eigenständig erkannt.
%           Für subsection der appendix kein autoref möglich!
%           > manuell über hyperref Kommando erledigen!
% Gebrauch: \autoref{[label]}
% Optionen: -
% vor/nach: -
% Referenz: $miktex$/doc/latex/hyperref/manual.pdf
%           $miktex$/doc/latex/hyperref/paper.pdf
%           $groups$/de.comp.text.tex/browse_thread/thread/21107c178bc2448e

% Für weitere Umgebungen müssen die Links entweder manuell gesetzt werden per:
% \hyperref[labelname]{Hier irgendein Text, siehe Bild \ref*{labelname}}
% oder: über z.B. \newcommand{\subfigureautorefname}{Bild} definiert werden

\AtBeginDocument{\renewcommand{\figureautorefname}{Abbildung}}
\AtBeginDocument{\newcommand{\subfigureautorefname}{Abbildung}}
\AtBeginDocument{\renewcommand{\tableautorefname}{Tabelle}}
\AtBeginDocument{\renewcommand{\equationautorefname}{Gleichung}}
\AtBeginDocument{\renewcommand{\chapterautorefname}{Kapitel}}
\AtBeginDocument{\renewcommand{\sectionautorefname}{Abschnitt}}
\AtBeginDocument{\renewcommand{\subsectionautorefname}{Abschnitt}}
\AtBeginDocument{\renewcommand{\subsubsectionautorefname}{Abschnitt}}
\AtBeginDocument{\renewcommand{\appendixautorefname}{Anhang}}
% \AtBeginDocument{\newcommand{\appendixsectionautorefname}{Anhang} > geht nur manuell!

%------------------------------------------------------------------------------%
%---- Literaturverzeichnis / Zitierweise---------------------------------------%
%------------------------------------------------------------------------------%
%\ExecuteBibliographyOptions{
%	sorting=,
%	sortcase=,
%	sortupper,
%	sortlocale=auto,
%	sortlos=bib,
%	related=true,
%	sortcites=true,
%	maxnames=,
%	minnames=,
%	maxbibnames=,
%	minbibnames=,
%	maxcitenames=,
%	mincitenames=,
%	maxitems=,
%	minitems=,
%	autocite=plain,
%	autopunct=true,
%	language=autobib,
%	clearlang=true,
%	autolang=none,
%	block=none,
%	notetype=foot+end,
%	hyperref=true,
%	backref=true,
%	backrefstyle=none,
%	backrefsetstyle=setonly,
%	indexing=true,
%	loadfiles=true,
%	refsection=none,
%	refsegment=none,
%	citereset=none,
%	abbreviate=true,
%	datelabel=year,
%	origdate=year,
%	eventdate=year,
%	urldate=year,
%	alldates=year,
%	datezeros=true,
%	dateabbrev=true,
%	defernumbers=true,
%	punctfont=true,
%	arxiv=abs,
%	texencoding=auto,
%	bibencoding=auto,
%	safeinputenc=true,
%	bibwarn=true,
%	mincrossrefs=,
%	isbn=true,
%	url=true,
%	doi=true,
%	eprint=true
%	}


\addbibresource{content/backmatter/literatur.bib}
%\cite[prefix][suffix]{ref}
%\parencite[prefix][suffix]{ref}
%\footcite[prefix][suffix]{ref}
%\textcite[prefix][suffix]{ref}
%\autocite[prefix][suffix]{ref}

\pgfplotsset{compat=1.8}